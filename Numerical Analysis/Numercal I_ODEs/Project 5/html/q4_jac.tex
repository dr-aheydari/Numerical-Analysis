
% This LaTeX was auto-generated from MATLAB code.
% To make changes, update the MATLAB code and republish this document.

\documentclass{article}
\usepackage{graphicx}
\usepackage{color}

\sloppy
\definecolor{lightgray}{gray}{0.5}
\setlength{\parindent}{0pt}

\begin{document}

    
    
\subsection*{Contents}

\begin{itemize}
\setlength{\itemsep}{-1ex}
   \item Interactive Interface (with user input)
   \item Non Interactive (without user input)
   \item Different norms and Tolerances
   \item Output Formatting
   \item Output
   \item Function
\end{itemize}
\begin{verbatim}
% Ali Heydari
% Math 231 Homework 4
% Jacobi Method (Q4) N = 50
\end{verbatim}


\subsection*{Interactive Interface (with user input)}

\begin{par}
get input A = input(?Please enter the matrix A: ?); x\_init = input(' Please enter a vector of initial guesses for x :'); delta = input(?Please enter the desired tolerance: ?);
\end{par} \vspace{1em}


\subsection*{Non Interactive (without user input)}

\begin{par}
making the tridiagonal matrix
\end{par} \vspace{1em}
\begin{verbatim}
n = 400;
A=full(gallery('tridiag',n,-1,2,-1));

b = zeros(n,1);
b(1) = 1;

% arbitrary initial guess
x_init = zeros(n,1);
n = size(x_init,1);
error = 1;



% different norms

% l2 norm
L = 2;
\end{verbatim}


\subsection*{Different norms and Tolerances}

\begin{par}
find the solution based on what norm and what error: delta = 10\^{}-2; [x\_sol\_l2\_1 ,counter\_l2\_1] = gauss\_seidel(A, x\_init, delta, error, n, b, L);
\end{par} \vspace{1em}
\begin{verbatim}
for i = 1 : 5
delta = 10^-i;
tic;
[x_sol_l2_2 ,counter_l2_2] = gauss_seidel(A, x_init, delta, error, n, b, L);
time = toc;

fprintf("iter: %i    time:%i \n",counter_l2_2,time);
end
% l_infty norm (denoted as l8)
% L = Inf;
% delta = 10^-2;
% [x_sol_l8_1 ,counter_l8_1] = gauss_seidel(A, x_init, delta, error, n, b, L);
%
% delta = 10^-5;
% [x_sol_l8_2 ,counter_l8_2] = gauss_seidel(A, x_init, delta, error, n, b, L);
\end{verbatim}


\subsection*{Output Formatting}

\begin{verbatim}
% fprintf(" APPROXIMATIONS: \n")
% disp(" ")
% fprintf(" For l-2 norm and 10^-2   |   For l-2 norm and 10^-5 | \n ");
% disp("----------------------------------------------------- ")
% for i = 1 : size(x_sol_l2_2)
%
%     fprintf("        %f          |         %f\n", x_sol_l2_2(i),x_sol_l2_2(i));
%
% end
% disp("----------------------------------------------------- ")
% fprintf(" NUMBER OF ITERATIONS\n ")
% disp(" ")
% fprintf("For l-2 norm and 10^-2 | For l-2 norm and 10^-5 |\n");
% disp("----------------------------------------------------- ")
%  fprintf("        %f          |         %f\n", counter_l2_1,counter_l2_2);
% disp(" ")
% fprintf(" APPROXIMATIONS: \n")
% fprintf(" For l-Inf norm and 10^-2  |  For l-Inf norm and 10^-5 | \n ");
% disp("----------------------------------------------------- ")
% for i = 1 : size(x_sol_l8_1)
%
%     fprintf("        %f           |         %f\n", x_sol_l8_1(i),x_sol_l8_2(i));
%
% end
%
% disp("----------------------------------------------------- ")
% disp(" ")
% fprintf(" NUMBER OF ITERATIONS\n ")
% fprintf("For l-Inf norm and 10^-2 | For l-Inf norm and 10^-5 |\n");
% disp("----------------------------------------------------- ")
%  fprintf("        %f          |         %f\n", counter_l8_1,counter_l8_2);
\end{verbatim}


\subsection*{Output}



\subsection*{Function}

\begin{verbatim}
function[x_sol,counter,time] = gauss_seidel(A, x_init, delta, error, n, b, L)
tic;
counter = 0;

% the algorithm
while error > delta

    x_old = x_init;

    for i = 1:n

        % this is essentially a place holder for the sum part
        AX = 0;

        % since i not= j, we do two for loops
        for j=1:n

            if j~=i
                AX = AX + A(i,j) * x_init(j);
            end

        end

        % now get the value of each
        x_init(i) = (1 / A(i,i)) * (b(i) - AX);

    end

    % update error and counter

    if L == Inf

        error = norm((x_old - x_init),Inf);

    end

    if L == 2

            error = norm((x_old - x_init));
    end

    counter = counter + 1;
end

% storing our answers
counter = counter - 1;
x_sol = x_init;
time = toc;
end
\end{verbatim}

        \color{lightgray} \begin{verbatim}iter: 4    time:2.156054e-02 
iter: 95    time:8.994348e-02 
iter: 2127    time:1.502804e+00 
iter: 27775    time:1.867436e+01 
iter: 65287    time:4.705305e+01 
\end{verbatim} \color{black}
    


\end{document}
    
