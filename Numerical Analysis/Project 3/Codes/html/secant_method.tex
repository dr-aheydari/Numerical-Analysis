
% This LaTeX was auto-generated from MATLAB code.
% To make changes, update the MATLAB code and republish this document.

\documentclass{article}
\usepackage{graphicx}
\usepackage{color}

\sloppy
\definecolor{lightgray}{gray}{0.5}
\setlength{\parindent}{0pt}

\begin{document}

    
    
\subsection*{Contents}

\begin{itemize}
\setlength{\itemsep}{-1ex}
   \item Interactive Interface (with user input)
   \item Non Interactive (without user input)
   \item Output Formatting
   \item Outputs
\end{itemize}
\begin{verbatim}
% Ali Heydari
% Math 231, hw3
% Secant Method


% Problem 3)part a) second root x = 1

x_k = zeros(1,10);
error = zeros(1,10);
e_n = zeros(1,10);
\end{verbatim}


\subsection*{Interactive Interface (with user input)}

\begin{verbatim}
% % get initial conditions
% x_k(1) = input('Please enter the initial guess x_0: ');
% x_k(2) = input('Please enter the initial guess x_1: ');
% delta = input('Please enter the desired tolerance: ');
% f = input('Please enter f(x)?(type @(x) [then the function] ');
\end{verbatim}


\subsection*{Non Interactive (without user input)}

\begin{verbatim}
f = @(x) x^3 - 3*x + 2;
% our two initial guesses
x_k(1) = 1.2;
x_k(2) = 0;
xk = x_k(1);
fx = f(x_k(1));
delta = 10^-6 ;

% evaluate function at x_o: f(x_o)
fx = f(x_k(2));

counter = 2; % counter

% keep finding the root until f(x_k) = 0
while abs(f(x_k(counter))) > delta

% secant method formula
x_k(counter+1) = x_k(counter) - (f(x_k(counter))*(x_k(counter) ...
    - x_k(counter-1)))/( f(x_k(counter)) - f(x_k(counter-1)));
% %display xk and f(xk)
 xk = x_k(counter+1);
 fx = f(x_k(counter+1));
% find error error:

e_n(counter) = abs(xk - (-2));
     if fx <= 1 * 10 ^ -6

            root = xk;
     end

error(counter) = abs(x_k(counter+1) - x_k(counter));
counter = counter + 1;

end
\end{verbatim}


\subsection*{Output Formatting}

\begin{verbatim}
disp(" ");
disp(" ");
fprintf('The root of the function is at x = %i \n', root);
fprintf('Number of iterations: %i \n', counter);
disp(" ");
disp(" ");

disp("        pn              |p_{n+1} - p_n|       e_n = |pn - p| ")
for i= 1 : counter - 1
    % to only output 10 values
    if i < 11
        fprintf("%i   %i         %i         %i\n",i ,x_k(i),error(i),e_n(i));
    end
end
\end{verbatim}


\subsection*{Outputs}
(Next Page)
\newpage
        \color{lightgray} \begin{verbatim} 
 
The root of the function is at x = 1.000427e+00 
Number of iterations: 18 
 
 
        pn              |p_{n+1} - p_n|       e_n = |pn - p| 
1   1.200000e+00         0                    0
2   0                    1.282051e+00         3.282051e+00
3   1.282051e+00         1.925003e-01         3.474552e+00
4   1.474552e+00         2.889026e-01         3.185649e+00
5   1.185649e+00         4.715544e-02         3.138494e+00
6   1.138494e+00         5.723392e-02         3.081260e+00
7   1.081260e+00         2.922022e-02         3.052039e+00
8   1.052039e+00         1.999151e-02         3.032048e+00
9   1.032048e+00         1.208593e-02         3.019962e+00
10   1.019962e+00         7.611857e-03         3.012350e+00
\end{verbatim} \color{black}
    


\end{document}
    
