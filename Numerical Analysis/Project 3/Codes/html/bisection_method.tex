
% This LaTeX was auto-generated from MATLAB code.
% To make changes, update the MATLAB code and republish this document.

\documentclass{article}
\usepackage{graphicx}
\usepackage{color}

\sloppy
\definecolor{lightgray}{gray}{0.5}
\setlength{\parindent}{0pt}

\begin{document}

    
    
\subsection*{Contents}

\begin{itemize}
\setlength{\itemsep}{-1ex}
   \item Interactive Interface (with user input)
   \item Non Interactive (without user input)
   \item Method
   \item Output Formatting
   \item Outputs
\end{itemize}
\begin{verbatim}
% Ali Heydari
% Math 231, hw3
% Bisection Method
\end{verbatim}


\subsection*{Interactive Interface (with user input)}

\begin{par}
\% get input a = input('Please enter a value for the lower bound a: '); b = input('Please enter a value for the upper bound (b) : '); delta = input('Please enter the desired tolerance: '); f = input('Please enter f(x)?(type @(x) [then the function] ');
\end{par} \vspace{1em}
\begin{verbatim}
% see if any of the boundaries are a root
\end{verbatim}


\subsection*{Non Interactive (without user input)}

\begin{verbatim}
f = @(x) x^3 - 3*x + 2;


retur = 0;
counter = 0;
x_k = ones(1,10);
error = zeros(1,10);
e_n = zeros(1,10);

delta = 10^-6
a = -4;
b = 0;
\end{verbatim}


\subsection*{Method}

\begin{verbatim}
fa = f(a);

if fa == 0

    root = a;
    retur = 1;

end;

fb = f(b);

if fb == 0
    root = b;
    retur = 1;

end;

% if the boundaries are not the root then do bisection

if retur ~= 1

    % check if the user hasnt lost their mind
    if sign(fa) == sign(fb)

        display('Error: f(a) and f(b) have same sign.')

        retur = 1;

    end;

    % if all is gucci


           while abs(b-a) > delta && retur ~= 1
                % As Mayya said in class, keep going until Iterate I <= 2delta

                counter = counter + 1;

                c = (a+b)/2;
                fc = f(c);

                if fc == 0
                        root = c;
                        retur = 1;
                end;

                % cehk to see which side of the interval we want

                if sign(fc) == sign(fa)
                        a = c;
                        fa = fc;
                else
                        b = c;
                        fb = fc;
                end;

            end;

            % hopefully we got want we need
            root = (a+b)/2;
end

fprintf('The found root is: %i \n',root);
fprintf('Total iterations: %i \n', counter);
\end{verbatim}


\subsection*{Output Formatting}

\begin{verbatim}
disp(" ");
disp(" ");
fprintf('The root of the function is at x = %i \n', root);
fprintf('Number of iterations: %i \n', counter);
disp(" ");
disp(" ");

disp("        pn              |p_{n+1} - p_n|       e_n = |pn - p| ")
for i= 1 : counter

fprintf("%i   %i         %i         %i\n",i ,x_k(i),error(i),e_n(i));

end
\end{verbatim}


\subsection*{Outputs}


        \color{lightgray} \begin{verbatim}
delta =

   1.0000e-06

The found root is: -2
Total iterations: 9 
 
 
The root of the function is at x = -2 
Number of iterations: 9 
 
 
        pn              |p_{n+1} - p_n|       e_n = |pn - p| 
       -2.1                 0.15                  0.1
       -1.95                 0.075                0.05
       -2.025                0.0375               0.025
       -1.9875               0.01875              0.0125
       -2.0063           0.009375         0.0062499
       -1.9969           0.0046875        0.0031251
       -2.0016           0.0023438        0.0015624
       -1.9992           0.0011719        0.00078135
       -2.0004           0.00058594       0.00039053 
\end{verbatim} \color{black}
    


\end{document}
    
